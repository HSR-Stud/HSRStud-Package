% \iffalse meta-comment
%<*internal>
\def\nameofplainTeX{plain}
\ifx\fmtname\nameofplainTeX\else
  \expandafter\begingroup
\fi
%</internal>
%<*install>
\input docstrip.tex
\keepsilent
\askforoverwritefalse
\usedir{tex/latex/hsrstud}
\generate{
  \file{\jobname.sty}{\from{\jobname.dtx}{package}}
}
%</install>
%<install>\endbatchfile
%<*internal>
\usedir{source/latex/hsrstud}
\generate{
  \file{\jobname.ins}{\from{\jobname.dtx}{install}}
}
\nopreamble\nopostamble
\ifx\fmtname\nameofplainTeX
  \expandafter\endbatchfile
\else
  \expandafter\endgroup
\fi
%</internal>


%<*package>
\NeedsTeXFormat{LaTeX2e}
\ProvidesPackage{hsrstud}[2020/04/16 v0.1 HSR-Stud Macros]

\RequirePackage{amsmath}
\RequirePackage{amssymb}
\RequirePackage{bm}

\RequirePackage{xcolor}
%</package>


%<*driver>
\documentclass[a4paper]{ltxdoc}

\usepackage{\jobname}
\usepackage[numbered]{hypdoc}

\EnableCrossrefs
\CodelineIndex
\RecordChanges

\begin{document}
  \DocInput{\jobname.dtx}
\end{document}
%</driver>
% \fi


% \GetFileInfo{\jobname.sty}
% \title{^^A
%   \texttt{hsrstud} --- HSR-Stud Style and Macros\thanks{^^A
%     This file describes version \fileversion, last revised \filedate.^^A
%   }^^A
% }
% \author{Naoki Pross \texttt{<npross@hsr.ch>}}
% \date{Released \filedate}
% \maketitle
% \changes{v1.0}{2020/02/16}{Initial Draft}


% \iffalse
%    \begin{macrocode}
%<*package>
%    \end{macrocode}
% \fi


% \section{Package Options}
%    \begin{macrocode}
\newif\if@arrowvec\@arrowvecfalse
\DeclareOption{arrowvec}{\@arrowvectrue}

\ProcessOptions\relax
%    \end{macrocode}



% \section{Mathematics}

% \subsection{Formatting}

% \begin{macro}{\vec}
% Typeset vectors, if the option 
%    \begin{macrocode}
\if@arrowvec
\else
\renewcommand{\vec}[1]{\ensuremath{\bm{#1}}}
\fi
%    \end{macrocode}
% \end{macro}


% \subsection{Equalities}

% \begin{macro}{\heq}
% L'H\^opital limit equality symbol.
%    \begin{macrocode}
\newcommand{\heq}{\stackrel{\hat{\texttt{H}}}{=}}
%    \end{macrocode}
% \end{macro}

% \subsection{Derivatives}

% \begin{macro}{\dd}
% The differential element. It needs a \marg{var} and has the optional argument \oarg{exponent}.
%    \begin{macrocode}
\newcommand{\dd}[2][]{\mathrm{d}^{#1} #2}
%    \end{macrocode}
% \end{macro}

% \begin{macro}{\di}
% This is the same as \cs{dd} but with a small space before it, it is intended to be used in integrals for a nicer typesetting.
%    \begin{macrocode}
\newcommand{\di}[2][]{\,\dd[#1]{#2}}
%    \end{macrocode}
% \end{macro}

% \begin{macro}{\deriv}
% The derivative has arguments \marg{function}, \marg{var} and the optional argument \oarg{order}.
%    \begin{macrocode}
\newcommand{\deriv}[3][]{\frac{\dd[#1]{#2}}{\dd[]{#3^{#1}}}}
%    \end{macrocode}
% \end{macro}

% \begin{macro}{\pderiv}
% The partial derivative has arguments \marg{function}, \marg{var} and the optional argument \oarg{order}.
%    \begin{macrocode}
\newcommand{\pderiv}[3][]{\frac{\partial^{#1} #2}{\partial^{#1} #3}}
%    \end{macrocode}
% \end{macro}

% \begin{macro}{\grad}
% The gradient operator.
%    \begin{macrocode}
\newcommand{\grad}{\nabla}
%    \end{macrocode}
% \end{macro}

% \begin{macro}{\div}
% The divergence operator, \cs{div} is renamed to \cs{divsymb}.
%    \begin{macrocode}
\let\divsymb=\div 
\renewcommand{\div}{\nabla\cdot}
%    \end{macrocode}
% \end{macro}


% \begin{macro}{\curl}
% The curl operator.
%    \begin{macrocode}
\newcommand{\curl}{\nabla\times}
%    \end{macrocode}
% \end{macro}

% \begin{macro}{\laplace}
% The laplace operator.
%    \begin{macrocode}
\newcommand{\laplace}{\nabla^2}
%    \end{macrocode}
% \end{macro}

% \section{Colors}
%    \begin{macrocode}
\definecolor{hsr-blue}{HTML}{0065A3}
\definecolor{hsr-blue80}{HTML}{3384B5}
\definecolor{hsr-blue60}{HTML}{66A3C8}
\definecolor{hsr-blue40}{HTML}{99C1DA}
\definecolor{hsr-blue20}{HTML}{CCE0ED}

\definecolor{hsr-mauve}{HTML}{6E1C50}
\definecolor{hsr-mauve80}{HTML}{8B4973}
\definecolor{hsr-mauve60}{HTML}{A87796}
\definecolor{hsr-mauve40}{HTML}{C5A4B9}
\definecolor{hsr-mauve20}{HTML}{E2D2DC}

\definecolor{hsr-lakegreen}{HTML}{548C86}
\definecolor{hsr-lakegreen80}{HTML}{76A39E}
\definecolor{hsr-lakegreen60}{HTML}{98BAB6}
\definecolor{hsr-lakegreen40}{HTML}{BBD1CF}
\definecolor{hsr-lakegreen20}{HTML}{DDE8E7}

\definecolor{hsr-reed}{HTML}{7B6951}
\definecolor{hsr-reed80}{HTML}{958774}
\definecolor{hsr-reed60}{HTML}{B0A597}
\definecolor{hsr-reed40}{HTML}{CAC3B9}
\definecolor{hsr-reed20}{HTML}{E5E1DC}

\definecolor{hsr-petrol}{HTML}{00738D}
\definecolor{hsr-petrol80}{HTML}{338FA4}
\definecolor{hsr-petrol60}{HTML}{66ABBB}
\definecolor{hsr-petrol40}{HTML}{99C7D1}
\definecolor{hsr-petrol20}{HTML}{CCE3E8}

\definecolor{hsr-basswood}{HTML}{BABD5D}
\definecolor{hsr-basswood80}{HTML}{C8CA7D}
\definecolor{hsr-basswood60}{HTML}{D6D79E}
\definecolor{hsr-basswood40}{HTML}{E3E5BE}
\definecolor{hsr-basswood20}{HTML}{F1F2DF}

\definecolor{hsr-lightgrey}{HTML}{C6C7C8}
\definecolor{hsr-lightgrey80}{HTML}{D1D2D3}
\definecolor{hsr-lightgrey60}{HTML}{DDDDDE}
\definecolor{hsr-lightgrey40}{HTML}{E8E8E9}
\definecolor{hsr-lightgrey20}{HTML}{F4F4F4}

\definecolor{hsr-black}{HTML}{1A171B}
\definecolor{hsr-black80}{HTML}{484549}
\definecolor{hsr-black60}{HTML}{767476}
\definecolor{hsr-black40}{HTML}{A4A2A4}
\definecolor{hsr-black20}{HTML}{D1D1D1}
%    \end{macrocode}

% \StopEventually{^^A
%   \PrintChanges
%   \PrintIndex
% }


% \iffalse
% \begin{macrocode}
%</package>
% \end{macrocode}
% \fi
%\Finale
