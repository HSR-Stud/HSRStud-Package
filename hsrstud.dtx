% \iffalse meta-comment
%<*internal>
\def\nameofplainTeX{plain}
\ifx\fmtname\nameofplainTeX\else
  \expandafter\begingroup
\fi
%</internal>
%<*install>
\input docstrip.tex
\keepsilent
\askforoverwritefalse
\usedir{tex/latex/hsrstud}
\generate{
  \file{\jobname.sty}{\from{\jobname.dtx}{package}}
}
%</install>
%<install>\endbatchfile
%<*internal>
\usedir{source/latex/hsrstud}
\generate{
  \file{\jobname.ins}{\from{\jobname.dtx}{install}}
}
\nopreamble\nopostamble
\ifx\fmtname\nameofplainTeX
  \expandafter\endbatchfile
\else
  \expandafter\endgroup
\fi
%</internal>


%<*package>
\NeedsTeXFormat{LaTeX2e}
\ProvidesPackage{hsrstud}[2020/04/16 v0.1 HSR-Stud Macros]

\RequirePackage{amsmath}
\RequirePackage{amssymb}
\RequirePackage{esint}
\RequirePackage{bm}

\RequirePackage{xcolor}
%</package>


%<*driver>
\documentclass[a4paper]{ltxdoc}

\usepackage{\jobname}

\usepackage[numbered]{hypdoc}
\hypersetup{
    colorlinks=true,
    linkcolor=black,
    citecolor=red!50!black,
    filecolor=magenta!50!black,
    urlcolor=blue!50!black
}

\usepackage{amsmath}
\usepackage{amssymb}

\usepackage{verbatim}
\usepackage{enumitem}
\usepackage{framed}
\usepackage{booktabs}
\usepackage{tikz}

\usepackage[
    type={CC},
    modifier={by-nc-sa},
    version={4.0},
]{doclicense}

\EnableCrossrefs
\CodelineIndex
\RecordChanges


\newenvironment{example}[1]{%
    \leftbar
    \begingroup\centering
    \minipage[c]{.4\linewidth}
        #1
    \endminipage%
    \minipage[c]{.4\linewidth}
}{%
    \endminipage
    \endgroup
    \endleftbar
}

\begin{document}
    \DocInput{\jobname.dtx}
\end{document}
%</driver>
% \fi


%    \GetFileInfo{\jobname.sty}
%    \title{^^A
%      \texttt{\textcolor{hsr-blue}{hsr}stud} --- HSR-Stud Style and Macros\thanks{^^A
%        This file describes version \fileversion, last revised \filedate.^^A
%      }^^A
%    }
%    \author{Naoki Pross \texttt{<npross@hsr.ch>}}
%    \date{Released \filedate}
%    \maketitle
%    \changes{v1.0}{2020/02/16}{Initial Draft}
%    \tableofcontents


%    \iffalse
%    \begin{macrocode}
%<*package>
%    \end{macrocode}
%    \fi


%    \section{Package Options} \label{sec:options}
%    \begin{description}[align=right]
%       \item[\texttt{arrowvec}] Tells the package to use a vector notation 
%       with a small arrow over the variables, as it were handwritten.
%
%       \item[\texttt{textvecdiff}] Disables the ``Nabla'' or ``Del'' notation 
%       for vector derivatives. Instead the symbols
%       \(\nabla, \nabla \cdot, \nabla \times, \nabla^2\) are be replaced with 
%       grad, div, curl and div grad.
%
%
%    \end{description}
%   
%    \DisableCrossrefs
%    \begin{macrocode}
\newif\if@arrowvec\@arrowvecfalse
\DeclareOption{arrowvec}{\@arrowvectrue}

\newif\if@textvecdiff\@textvecdifffalse
\DeclareOption{textvecdiff}{\@textvecdifftrue}

\ProcessOptions\relax
%    \end{macrocode}
%    \EnableCrossrefs


%    \section{Mathematics}
%    \subsection{Formatting}

%    \begin{macro}{\vec}
%    Vectors notation. If the option \texttt{arrowvec} described in \S 
%    \ref{sec:options} is enabled, the notation with a small arrow over the 
%    varible will be used, else the vector is bold. Takes one option 
%    \marg{letter}.
%    \begin{example}{\[\vec{F} = m\vec{a}\]}
%        |\vec{F} = m\vec{a}|
%    \end{example}
%
%    \begin{macrocode}
\if@arrowvec
\else
\renewcommand{\vec}[1]{\mathbf{\boldsymbol#1}}
\fi
%    \end{macrocode}
%    \end{macro}

%    \begin{macro}{\uvec}
%    Unit vector notation. Takes \marg{letter}.
%    \begin{example}{\[\uvec{x} = \vec{x}/x\]}
%        |\uvec{x} = \vec{x}/x|
%    \end{example}
%
%    \begin{macrocode}
\newcommand{\uvec}[1]{\mathrm{\vec{\hat{#1}}}}
%    \end{macrocode}
%    \end{macro}

%    \begin{macro}{\mtx}
%    Matrix notation. Takes \marg{letter}.
%    \begin{example}{\[
%        \mtx{J} = \begin{pmatrix} 0 & 1 \\ 1 & 0 \end{pmatrix}
%    \]}^^A
%        |\mtx{J} = \begin{pmarix} |\\
%        |            0 & 1  \\    |\\
%        |            1 & 0        |\\
%        |          \end{pmatrix}  |
%    \end{example}
%
%    \begin{macrocode}
\newcommand{\mtx}[1]{\mathrm{#1}}
%    \end{macrocode}
%    \end{macro}

%    \begin{macro}{\ten}
%    Tensor notation. Takes \marg{letter}.
%    \begin{example}{\[
%        \vec{T}^{(\vec{n})} = \uvec{n}\cdot\ten{\sigma}
%    \]}^^A
%        |\vec{T}^{(\vec{n})} =        |\\
%        |    \uvec{n}\cdot\ten{\sigma}|
%    \end{example}
%    \begin{macrocode}
\newcommand{\ten}[1]{\underline{\mathbf{\boldsymbol{#1}}}}
%    \end{macrocode}
%    \end{macro}


%    \subsection{Equalities}

%    \begin{macro}{\heq}
%     L'H\^opital limit equality symbol.
%    \begin{example}{\[
%        \lim_{x\to\infty} \frac{x}{x^2 - 1}
%            \heq \lim_{x\to\infty} \frac{1}{2x}
%            = 0
%    \]}
%        |\lim_{x\to\infty} \frac{x}{x^2 - 1}   |\\
%        |  \heq \lim_{x\to\infty} \frac{1}{2x} |\\
%        |  = 0 |
%    \end{example}

%    \begin{macrocode}
\newcommand{\heq}{\stackrel{\hat{\texttt{H}}}{=}}
%    \end{macrocode}
%    \end{macro}

%    \subsection{Derivatives}

%    \begin{macro}{\dd}
%    The differential element. It needs a \marg{var} and has the optional 
%    argument \oarg{exponent}.
%    \begin{example}{\[\dd{x}\qquad\dd[4]{x}\]}|\dd{x} \dd[4]{x}|\end{example}
%    
%    \begin{macrocode}
\newcommand{\dd}[2][]{\mathrm{d}^{#1} #2}
%    \end{macrocode}
%    \end{macro}

%    \begin{macro}{\di}
%    This is the same as \cs{dd} but with a small space in front, it is 
%    intended to be used in integrals for a nicer typesetting.
%    \begin{example}{\begin{align*}
%        I &= \int \vec{J}\cdot\dd{\vec{s}} \\
%          &= \iint \vec{J}\cdot\uvec{n}\di{x}\di{y}
%    \end{align*}}
%        |% no spacing needed (because of cdot)| \\
%        |I = \int\vec{J}\cdot\di{\vec{s}}     | \\
%        |% needs spacing                      | \\
%        |  = \iint\vec{J}\cdot\uvec{n}\di{x}\di{y} |
%    \end{example}
% 
%    \begin{macrocode}
\newcommand{\di}[2][]{\,\dd[#1]{#2}}
%    \end{macrocode}
%    \end{macro}

%    \begin{macro}{\deriv}
%    The derivative has arguments \marg{function}, \marg{var} and the optional 
%    argument \oarg{order}.
%    \begin{example}{\[\deriv{y}{x} \qquad \deriv[3]{y}{x}\]}
%        |\deriv{y}{x}    |\\
%        |\deriv[3]{y}{x} |
%    \end{example}
%
%    \begin{macrocode}
\newcommand{\deriv}[3][]{\frac{\dd[#1]{#2}}{\dd[]{#3^{#1}}}}
%    \end{macrocode}
%    \end{macro}

%    \begin{macro}{\pderiv}
%    The partial derivative has arguments \marg{function}, \marg{var} and the 
%    optional argument \oarg{order}.
%    \begin{example}{\[\pderiv{y}{x} \qquad \pderiv[2]{y}{x}\]}
%        |\deriv{f}{x}    |\\
%        |\deriv[2]{f}{x} |
%    \end{example}
%
%    \begin{macrocode}
\newcommand{\pderiv}[3][]{\frac{\partial^{#1} #2}{\partial #3^{#1}}}
%    \end{macrocode}
%    \end{macro}

%    \begin{macro}{\grad}
%    The gradient operator.
%    \begin{example}{\[\grad f\]}|\grad f|\end{example}
%
%    \begin{macrocode}
\if@textvecdiff
    \newcommand{\grad}{\text{grad }}
\else
    \newcommand{\grad}{\nabla}%
\fi
%    \end{macrocode}
%    \end{macro}

%    \begin{macro}{\div}
%    The divergence operator, \cs{div} is renamed to \cs{divsymb}.
%    \begin{example}{\[\div f\]}|\div f|\end{example}
%
%    \begin{macrocode}
\let\divsymb=\div 
\if@textvecdiff
    \renewcommand{\div}{\text{div}}
\else
    \renewcommand{\div}{\nabla\cdot}
\fi
%    \end{macrocode}
%    \end{macro}


%    \begin{macro}{\curl}
%    The curl operator.
%    \begin{example}{\[\curl f\]}|\curl f|\end{example}
%
%    \begin{macrocode}
\if@textvecdiff
    \newcommand{\curl}{\text{curl }}
\else
    \newcommand{\curl}{\nabla\times}
\fi
%    \end{macrocode}
%    \end{macro}

%    \begin{macro}{\laplace}
%    The laplace operator.
%    \begin{example}{\[\laplace f\]}|\laplace f|\end{example}
%
%    \begin{macrocode}
\if@textvecdiff
    \newcommand{\laplace}{\text{div grad}}
\else
    \newcommand{\laplace}{\nabla^2}
\fi
%    \end{macrocode}
%    \end{macro}

%    \section{Colors}
%    \begin{center}
%    \begin{tikzpicture}
%       \foreach \color/\y in {%
%           hsr-blue/7,
%           hsr-mauve/6,
%           hsr-lakegreen/5,
%           hsr-reed/4,
%           hsr-petrol/3,
%           hsr-basswood/2,
%           hsr-lightgrey/1,
%           hsr-black/0}{
%           \foreach \x/\a in {0/,1/80,2/60,3/40,4/20}{
%               \node[anchor=east] at (-.5,\y) {\texttt{\color}};
%               \fill[color=\color\a] (\x,\y-.25) rectangle ++(1,.5);
%               \node[anchor=west] at (\x,\y) {\texttt{\a}};
%           }
%       }
%    \end{tikzpicture}
%    \end{center}

%    \StopEventually{^^A
%       \PrintChanges
%       \PrintIndex
%
%       \section{License}
%       \doclicenseThis
%   }
%
%    \DisableCrossrefs
%    \begin{macrocode}
\definecolor{hsr-blue}{HTML}{0065A3}
\definecolor{hsr-blue80}{HTML}{3384B5}
\definecolor{hsr-blue60}{HTML}{66A3C8}
\definecolor{hsr-blue40}{HTML}{99C1DA}
\definecolor{hsr-blue20}{HTML}{CCE0ED}

\definecolor{hsr-mauve}{HTML}{6E1C50}
\definecolor{hsr-mauve80}{HTML}{8B4973}
\definecolor{hsr-mauve60}{HTML}{A87796}
\definecolor{hsr-mauve40}{HTML}{C5A4B9}
\definecolor{hsr-mauve20}{HTML}{E2D2DC}

\definecolor{hsr-lakegreen}{HTML}{548C86}
\definecolor{hsr-lakegreen80}{HTML}{76A39E}
\definecolor{hsr-lakegreen60}{HTML}{98BAB6}
\definecolor{hsr-lakegreen40}{HTML}{BBD1CF}
\definecolor{hsr-lakegreen20}{HTML}{DDE8E7}

\definecolor{hsr-reed}{HTML}{7B6951}
\definecolor{hsr-reed80}{HTML}{958774}
\definecolor{hsr-reed60}{HTML}{B0A597}
\definecolor{hsr-reed40}{HTML}{CAC3B9}
\definecolor{hsr-reed20}{HTML}{E5E1DC}

\definecolor{hsr-petrol}{HTML}{00738D}
\definecolor{hsr-petrol80}{HTML}{338FA4}
\definecolor{hsr-petrol60}{HTML}{66ABBB}
\definecolor{hsr-petrol40}{HTML}{99C7D1}
\definecolor{hsr-petrol20}{HTML}{CCE3E8}

\definecolor{hsr-basswood}{HTML}{BABD5D}
\definecolor{hsr-basswood80}{HTML}{C8CA7D}
\definecolor{hsr-basswood60}{HTML}{D6D79E}
\definecolor{hsr-basswood40}{HTML}{E3E5BE}
\definecolor{hsr-basswood20}{HTML}{F1F2DF}

\definecolor{hsr-lightgrey}{HTML}{C6C7C8}
\definecolor{hsr-lightgrey80}{HTML}{D1D2D3}
\definecolor{hsr-lightgrey60}{HTML}{DDDDDE}
\definecolor{hsr-lightgrey40}{HTML}{E8E8E9}
\definecolor{hsr-lightgrey20}{HTML}{F4F4F4}

\definecolor{hsr-black}{HTML}{1A171B}
\definecolor{hsr-black80}{HTML}{484549}
\definecolor{hsr-black60}{HTML}{767476}
\definecolor{hsr-black40}{HTML}{A4A2A4}
\definecolor{hsr-black20}{HTML}{D1D1D1}
%    \end{macrocode}
%    \EnableCrossrefs

%    \iffalse
%    \begin{macrocode}
%</package>
%    \end{macrocode}
%    \fi
%
%\Finale
