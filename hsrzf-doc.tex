% !TeX program = xelatex
% !TeX encoding = utf8
% !TeX root = hsrzf-doc.tex

\documentclass[concrete, header, margin=huge]{hsrzf}

%% packages
\usepackage{hsrstud}
\usepackage[
    type={CC},
    modifier={by-nc-sa},
    version={4.0},
]{doclicense}

\usepackage{booktabs}
\usepackage{enumitem}


%% Metadata
\title{\textcolor{hsr-blue}{HSR}--Stud Zusammenfassung Class}
\author{Naoki Pross \texttt{<npross@hsr.ch>}}
\date{\today}

%% Macros
\newcommand{\vb}[1]{\texttt{#1}}
\newcommand{\cs}[1]{\texttt{\textbackslash #1}}

%% Document
\begin{document}
\maketitle
\tableofcontents

\section{Purpose of this document class}
The following document class was made by and for the HSR-Stud organization. This class is intended to be used to create beautiful summaries and cheat sheets.

\section{Class Options}
\begin{description}[align=right]
  \item[\tt sans] Use Sans-Serif fonts.
  \item[\tt concrete] Use the ``Concrete Roman'' and ``Concrete Math'' fonts.
  \item[\tt header] Use the HSR--Stud standard header. The option {\tt noheader} disables the header.
  \item[\tt margin=<size>] Sets up margins to reasonable size.
  \item[\tt *] Any other given option is passed to the underlying class (article).
\end{description}

\begin{table}\centering
\begin{tabular}{l r r l r r}
  \toprule
  Size & H & V & Size & H & V \\
  \midrule
  huge    & 45 & 40 & small   & 20 & 30 \\
  large   & 35 & 35 & tiny    & 10 & 20 \\
  big     & 30 & 30 & minimal &  5 & 20 \\
  normal  & 25 & 30 \\
  \bottomrule
\end{tabular}
\caption{Sizes for the {\tt margin} option. All sizes are in millimeters}
\end{table}


\section{Extra Metadata}
Thanks to the {\tt titling} package, normal metadata informations are available throug the following commands: \cs{theauthor}, \cs{thedate}, \cs{thetitle}. Additionally, more metadata variables have been created for this class, to conserve informations related to the university course.
\begin{description}[align=right]
\item[\cs{course}] The degree programme of the discussed subject. Example: \vb{Elektrotechnik}
\item[\cs{module}] Short name of the module according to the class schedule. Example: \vb{ComEng1}
\item[\cs{semester}] Semester during which the summarized subject was taught. Example: \vb{FS20}
\item[\cs{institute}] Internal institute in which the module was taught. This field is optional. Example: \vb{ICOM}
\end{description}


\section{License}
\doclicenseThis

\end{document}
